\begin{frame}{5.2: executive summary}
\alert{Definitions:} one-to-one, onto, invertible, isomorphism.
\bspace
\alert{Procedures:} none.
\bspace
\alert{Theorems:} $T\colon V\rightarrow W$ is an isomorphism iff $\dim V=\dim W$ and either (i) $T$ is one-to-one, or (ii) $T$ is onto.
\end{frame}
\begin{frame}{Composition of linear transformations}
\footnotesize
\begin{theorem}
Suppose we have 
\[
V\xrightarrow[]{T_1} W\xrightarrow[]{T_2}U
\]
where $T_1$ and $T_2$ are both linear transformations. Then their composition 
\begin{eqnarray*}
T=T_2\circ T_1\colon V&\rightarrow& U\\
\boldv&\mapsto& T(\boldv)=T_2(T_1(\boldv)). 
\end{eqnarray*}
is a linear transformation; i.e., the composition of linear functions is linear. 
\end{theorem}
\pause
\begin{example}
Let $B\in M_{mn}$ and $A\in M_{rm}$, and consider the corresponding linear transformations
\[
\R^n\xrightarrow[]{T_B} \R^m\xrightarrow[]{T_A}\R^r.
\]
\pause
Claim: $T_A\circ T_B=T_{AB}$. \\
\pause Proof: for any $\boldx\in\R^n$ we have 
\[
T_A\circ T_B(\boldx)=T_A(T_B(\boldx))\pause=T_A(B\boldx)\pause=A(B\boldx)\pause=AB(\boldx)\pause=T_{AB}(\boldx).
\]
Thus $T_A\circ T_B=T_{AB}$. 
\end{example}
\end{frame}
\begin{frame}{Inverses}
\footnotesize
\begin{definition}
A linear transformation $T\colon V\rightarrow W$ is {\bf invertible} if there is a function 
$S\colon W\rightarrow V$ such that
\bb[(i)]
\ii $S\circ T=I_V$ (the identity function on $V$)
\ii $T\circ S=I_W$(the identity function on $W$).
\ee
In this case the function $S$ is called the {\bf inverse} of $T$, and is denoted $S=T^{-1}$. 
\bpause
From general function theory we know that $T$ has an inverse if and only if 
\bb
\ii $T$ is {\bf one-to-one} (i.e., $T(\boldv_1)=T(\boldv_2)\Rightarrow \boldv_1=\boldv_2$), and  
\ii $T$ is {\bf onto} (i.e., for all $\boldw\in W$ there is a $\boldv\in V$ such that $T(\boldv)=\boldw$). 
\ee 
\end{definition}
\end{frame}
\begin{frame}{Proof techniques}
\footnotesize
We currently have two different ways of proving that a given $T\colon V\rightarrow W$ is invertible. 
\bb
\ii Show directly that there is an inverse function $T^{-1}\colon W\rightarrow V$ satisfying $T^{-1}\circ T=I_V$ and $T\circ T^{-1}=I_W$. 
\ii Show directly that $T$ is one-to-one and onto. 
\ee
\pause
\alert{Example}. Let $A\in M_{nn}$. Prove $T_A\colon\R^n\rightarrow\R^n$ is invertible if and only if $A$ is invertible. 
\\
\pause Proof: \\
\alert{($\Leftarrow$)} Suppose $A$ is invertible, with inverse $A^{-1}$. Claim: $T_A^{-1}=T_{A^{-1}}$. Indeed we have 
\begin{eqnarray*}
T_{A^{-1}}(T_A(\boldx))\pause&=&T_{A^{-1}}(A\boldx)\pause=A^{-1}A\boldx=\boldx\\
\pause T_{A}(T_{A^{-1}}(\boldx))&=&T_{A}(A^{-1}\boldx)=AA^{-1}\boldx=\boldx
\end{eqnarray*}
\pause Thus $T_{A^{-1}}\circ T_A=I_{\R^n}$ and $T_A\circ T_{A^{-1}}=I_{\R^n}$, as desired. 
\bpause
\alert{($\Rightarrow$)} Suppose $T_{A}$ is invertible. Then $T_A$ is one-to-one. In particular, this means $\NS(T_A)=\{\boldzero\}$, since $T(\boldv)=T(\boldzero)\Rightarrow \boldv=\boldzero$. But $\NS(T_A)=\NS(A)$. Thus $\NS(A)=\{\boldzero\}$, which implies $A$ is invertible. 
\end{frame}
\begin{frame}
\footnotesize
\begin{theorem}
Let $T\colon V\rightarrow W$ be linear. 
\bb[(a)]
\ii If $T$ is invertible, then $T^{-1}\colon W\rightarrow V$ is linear; i.e., inverses of linear transformations are linear. 
\pause
\ii $T$ is invertible if and only if $\NS(T)=\{\boldzero_V\}$ and $\range(T)=W$. 
\ee 
\pause In particular if $V$ is finite-dimensional with $\dim(V)=n$, then the following are equivalent:
\bb[(i)]
\ii $T$ is invertible;
\ii $\dim V=\dim W=n$ and $\dim(\NS(T))=0$ (i.e., $\nullity(T)=0)$);
\ii $\dim V=\dim W=n$ and $\dim(\range(T))=n$ (i.e., $\nullity(T)=n$).
\ee
\end{theorem}
\end{frame}

\begin{frame}{Isomorphisms}
\footnotesize
\begin{definition}
An invertible linear transformation $T\colon V\rightarrow W$ is called an ${\bf isomorphism}$. 
\\
Two vector spaces $V$ and $W$ are called {\bf isomorphic} if there exists an isomorphism $T\colon V\rightarrow W$. 
\end{definition}
\pause To say that $V$ and $W$ are isomorphic is to say that $V$ and $W$ are \alert{essentially the same thing}, at least as far as vector spaces are concerned. 

\bpause 
Why? The invertible functions $T\colon V\rightarrow W$ and $T^{-1}\colon W\rightarrow V$ allow us to \alert{translate} any vector space property of $V$ into a vector space property of $W$ and vice versa. For example: 
\begin{eqnarray*}
\{\boldv_1,\dots , \boldv_r\} \text{ spans } V &\xLeftrightarrow[\ T^{-1} \ ]{\ T \ }& \{T(\boldv_1),\dots , T(\boldv_r)\} \text{ spans } W\\ \\
\{\boldv_1,\dots , \boldv_r\} \text{ is independent in } V &\xLeftrightarrow[\ T^{-1} \ ]{\ T \ }& \{T(\boldv_1),\dots , T(\boldv_r)\} \text{ is independent in } W\\ \\
\dim(V)&=&\dim(W)
\end{eqnarray*}  

\end{frame}
\begin{frame}
\begin{theorem}
Let $V$ be finite-dimensional, and let $S=\{\boldv_1,\dots ,\boldv_n\}$ be a basis. Then the coordinate map 
\begin{eqnarray*}
T=(\underline{\hspace{5pt}})_S\colon V&\rightarrow& \R^n\\
\boldv=k_1\boldv_1+\cdots +k_n\boldv_n&\mapsto&(\boldv)_S=(k_1,k_2,\dots,k_n)
\end{eqnarray*}
is an isomorphism. Thus $V$ is isomorphic to $\R^n$. 
\end{theorem}
\pause
\alert{Comment}. This theorem is kind of a bummer as it means that all of our so-called exotic vector spaces $M_{mn}, P_n, etc.$ are really just some $\R^n$ dressed up in a funny outfit. 
\bpause 
But hold on! Not \alert{all} of our exotic spaces. Just the \alert{finite-dimensional} ones. Thus $P_{\infty}, C^\infty(-\infty,\infty), C(-\infty,\infty), $ etc., remain exotic. 
\end{frame}